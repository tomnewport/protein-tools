%%%%%%%%%%%%%%%%%%%%%%%%%%%%%%%%%%%%%%%%%
% Stylish Article
% LaTeX Template
% Version 2.0 (13/4/14)
%
% This template has been downloaded from:
% http://www.LaTeXTemplates.com
%
% Original author:
% Mathias Legrand (legrand.mathias@gmail.com)
%
% License:
% CC BY-NC-SA 3.0 (http://creativecommons.org/licenses/by-nc-sa/3.0/)
%
%%%%%%%%%%%%%%%%%%%%%%%%%%%%%%%%%%%%%%%%%

%----------------------------------------------------------------------------------------
%	PACKAGES AND OTHER DOCUMENT CONFIGURATIONS
%----------------------------------------------------------------------------------------

\documentclass[fleqn,10pt]{SelfArx} % Document font size and equations flushed left

%----------------------------------------------------------------------------------------
%	COLUMNS
%----------------------------------------------------------------------------------------

\setlength{\columnsep}{0.55cm} % Distance between the two columns of text
\setlength{\fboxrule}{0.75pt} % Width of the border around the abstract

%----------------------------------------------------------------------------------------
%	COLORS
%----------------------------------------------------------------------------------------

\definecolor{color1}{RGB}{0,0,90} % Color of the article title and sections
\definecolor{color2}{RGB}{0,20,20} % Color of the boxes behind the abstract and headings

%----------------------------------------------------------------------------------------
%	HYPERLINKS
%----------------------------------------------------------------------------------------

\usepackage{hyperref} % Required for hyperlinks
\hypersetup{hidelinks,colorlinks,breaklinks=true,urlcolor=color2,citecolor=color1,linkcolor=color1,bookmarksopen=false,pdftitle={Title},pdfauthor={Author}}

%----------------------------------------------------------------------------------------
%	ARTICLE INFORMATION
%----------------------------------------------------------------------------------------

\JournalInfo{DTC Dissertation, Thomas Newport, Systems Biology 2013} % Journal information
\Archive{Rotation 2} % Additional notes (e.g. copyright, DOI, review/research article)

\PaperTitle{Article Title} % Article title

\Authors{Tom Newport} % Authors

\Keywords{} % Keywords - if you don't want any simply remove all the text between the curly brackets
\newcommand{\keywordname}{Keywords} % Defines the keywords heading name

%----------------------------------------------------------------------------------------
%	ABSTRACT
%----------------------------------------------------------------------------------------

\Abstract{There's nothing here yet}

%----------------------------------------------------------------------------------------

\begin{document}

\flushbottom % Makes all text pages the same height

\maketitle % Print the title and abstract box

\tableofcontents % Print the contents section

\thispagestyle{empty} % Removes page numbering from the first page

%----------------------------------------------------------------------------------------
%	ARTICLE CONTENTS
%----------------------------------------------------------------------------------------
% Total word count should be approx. 4500 words

% Add command to format binomial species name:
\newcommand{\bn}
[1]{\textit{#1}}

% Add shortcut for P. falciparum
\newcommand{\pf}
{\bn{P. falciparum }}

\section{Introduction} 
%Introduce P.falciparum, lives in RBCs. Pros and cons

The apicomplexan parasite \bn{Plasmodium falciparum} is the most virulent causative agent of malaria, and responsible for over 600,000 deaths annually \cite{WorldHealthOrganisation2013}. Along with other members of the \bn{Plasmodium} family, \pf has a complex lifecycle, moving between several different tissues in both mammalian and arthropod hosts. Symptomatic disease in humans occurs when \pf undergoes rounds of asexual reproduction inside human red blood cells (RBCs) \cite{Chen2000}.

In some respects, the intracellular environment of a red blood cell is an ideal location for parasite proliferation. The cells' lack of an MHC (Major Histocompatibility Complex) system, which would otherwise be used to identify intracellular pathogens to the host immune system, renders parasites immunologically invisible \cite{Kirchgatter2005}, whilst the vascular system allows the parasite to travel throughout the body.

The highly specialised nature of RBCs, however, means that the intracellular environment also presents significant challenges to parasite survival, and so \pf exports a range of proteins to radically transform the intracellular environment to one suitable for parasite proliferation.

Recent estimates suggest that at least 10\% of the \pf genome is

% Maurer's cleft and friends

\subsection*{The Plasmodium Export Element (PEXEL) Motif}
%The pexel motif
%How this helps export
\subsection*{PEXEL Negative Exported Proteins (PNEPs)}
%TODO
\subsection{Protein Structure Prediction}

\subsubsection*{Disorder Prediction (metaPrDOS}

\subsubsection*{Coiled Coil Prediction (Coils)}

\subsubsection*{Transmembrane Prediction (TMHMM)}

\subsubsection*{Combined Approaches (Phyre2 and InterPro)}


\section{Software Implementation}
%2000 words
\subsection{Automating Sequence Submission}
\subsection{Collating output formats}
\subsection{Visualisation}
\section{Limitations and Further Work}
%300 words
%------------------------------------------------
\phantomsection
\section*{Acknowledgments}

\addcontentsline{toc}{section}{Acknowledgments} % Adds this section to the table of contents

%----------------------------------------------------------------------------------------
%	REFERENCE LIST
%----------------------------------------------------------------------------------------
\phantomsection
\bibliographystyle{unsrt}
\bibliography{references}

%----------------------------------------------------------------------------------------

\end{document}