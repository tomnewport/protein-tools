%%%%%%%%%%%%%%%%%%%%%%%%%%%%%%%%%%%%%%%%%
% Stylish Article
% LaTeX Template
% Version 2.0 (13/4/14)
%
% This template has been downloaded from:
% http://www.LaTeXTemplates.com
%
% Original author:
% Mathias Legrand (legrand.mathias@gmail.com)
%
% License:
% CC BY-NC-SA 3.0 (http://creativecommons.org/licenses/by-nc-sa/3.0/)
%
%%%%%%%%%%%%%%%%%%%%%%%%%%%%%%%%%%%%%%%%%

%----------------------------------------------------------------------------------------
%	PACKAGES AND OTHER DOCUMENT CONFIGURATIONS
%----------------------------------------------------------------------------------------

\documentclass[fleqn,10pt]{SelfArx} % Document font size and equations flushed left

%----------------------------------------------------------------------------------------
%	COLUMNS
%----------------------------------------------------------------------------------------

\setlength{\columnsep}{0.55cm} % Distance between the two columns of text
\setlength{\fboxrule}{0.75pt} % Width of the border around the abstract

%----------------------------------------------------------------------------------------
%	COLORS
%----------------------------------------------------------------------------------------

\definecolor{color1}{RGB}{0,0,90} % Color of the article title and sections
\definecolor{color2}{RGB}{0,20,20} % Color of the boxes behind the abstract and headings

%----------------------------------------------------------------------------------------
%	HYPERLINKS
%----------------------------------------------------------------------------------------

\usepackage{hyperref} % Required for hyperlinks
\hypersetup{hidelinks,colorlinks,breaklinks=true,urlcolor=color2,citecolor=color1,linkcolor=color1,bookmarksopen=false,pdftitle={Title},pdfauthor={Author}}

%----------------------------------------------------------------------------------------
%	ARTICLE INFORMATION
%----------------------------------------------------------------------------------------

\JournalInfo{DTC Dissertation, Thomas Newport, Systems Biology 2013} % Journal information
\Archive{Rotation 2} % Additional notes (e.g. copyright, DOI, review/research article)

\PaperTitle{Article Title} % Article title

\Authors{Tom Newport} % Authors

\Keywords{} % Keywords - if you don't want any simply remove all the text between the curly brackets
\newcommand{\keywordname}{Keywords} % Defines the keywords heading name

%----------------------------------------------------------------------------------------
%	ABSTRACT
%----------------------------------------------------------------------------------------

\Abstract{There's nothing here yet}

%----------------------------------------------------------------------------------------

\begin{document}

\flushbottom % Makes all text pages the same height

\maketitle % Print the title and abstract box

\tableofcontents % Print the contents section

\thispagestyle{empty} % Removes page numbering from the first page

%----------------------------------------------------------------------------------------
%	ARTICLE CONTENTS
%----------------------------------------------------------------------------------------
% Total word count should be approx. 4500 words

% Add command to format binomial species name:
\newcommand{\bn}
[1]{\textit{#1}}

% Add shortcut for P. falciparum
\newcommand{\pf}
{\bn{P. falciparum }}

% Add semantic strong (replaces bold)
\newcommand{\str}
[1]{\textbf{#1}}

\section{Introduction} 

%% TN Introduce Plasmodium falciparum, establish that it lives inside red blood cells
The apicomplexan parasite \bn{Plasmodium falciparum} is the most virulent causative agent of malaria, and responsible for over 600,000 deaths annually \cite{WorldHealthOrganisation2013}. Along with other members of the \bn{Plasmodium} family, \pf has a complex lifecycle, moving between several different tissues in both mammalian and arthropod hosts. Symptomatic disease in humans occurs when \pf undergoes rounds of asexual reproduction inside human red blood cells (RBCs) \cite{Chen2000}.

%% TN Introduce RBC intracellular environment, benefits and challenges
In some respects, the intracellular environment of a red blood cell is an ideal location for parasite proliferation. The cells' lack of an MHC (Major Histocompatibility Complex) system, which would otherwise be used to identify intracellular pathogens to the host immune system, renders parasites immunologically invisible \cite{Kirchgatter2005}, whilst the vascular system allows the parasite to travel throughout the body. The highly specialised nature of RBCs, however, means that the intracellular environment also presents significant challenges to parasite survival.

%% TN Introduce challenges to pf survival
Mature red blood cells lack protein production and export machinery, and are a nutritionally poor environment, with a proteome dominated by haemoglobin, which typically accounts for around 98\% of the protein content of the cell \cite{DAlessandro2010}. \pf is able to digest RBC proteins, however haemoglobin lacks several amino acids required for protein production. Red blood cells are also  subject to regular 'quality control' in the spleen, where damaged or infected cells are killed and recycled \cite{Elsworth2014}.

%% TN Introduce exportome
In order to survive and proliferate inside RBCs, \pf exports a range of proteins which radically transform the red blood cell, collectively termed the \str{exportome}. Many of these proteins are involved in setting up a parasite-derived protein export system, capable of directing exported \pf proteins to sites both inside and outside the RBC. \pf resides inside a parasitophorous vacuole, and exports proteins via structures termed Maurer's Clefts \cite{Marti2013}, which bear some similarity to golgi apparatus \cite{Mundwiler-Pachlatko2013}.

%% TN Split paragraph
In order to avoid detection in the spleen, many exported proteins are associated with the formation of knobs, specialised structures which form at the RBC membrane and promote cytoadherence to epithelial cells, platelets and other red blood cells \cite{Kraemer2006}. Severe forms of malaria, including cerebral malaria, are believed to be caused by sequestration of infected red blood cells in deep tissues, as well as overinduction of inflammatory cytokines \cite{Chen2000}. Other exported components have been associated with increasing RBC membrane permeability to facilitate nutrient and waste exchange and strengthening the RBC cytoskeleton (Reviewed in \cite{Elsworth2014}). 

%% TN Introduce exportome in a genomic context
To date, at least 10\% of the protein products of the \pf genome have been shown to be exported to the host cell \cite{Boddey2013a}. Within this exportome, there exist 360 distinct proteins once close duplicates are excluded.


\subsection*{The Plasmodium Export Element (PEXEL) Motif}
%% Why predict the exportome?
Whilst the \pf genome has been available since 2002 \cite{Gardner2002}, comparatively few genes have been studied in depth, and many remain of unknown function. The discovery of a motif termed the PEXEL (Plasmodium Export Element) shared between many exportome components has made it possible to reliably predict the \pf exportome in the absence of other information \cite{Sargeant2006}.

%% The pexel motif, and how this is exported
The PEXEL motif is pentameric, located near the N-terminal of the protein, and can be generalised as the amino acid sequence \texttt{RxLxE/Q/D} \cite{Goldberg2010}, where \texttt{x} is any non-charged amino acid \cite{Dietz2014} although the non-canonical PEXEL motif \texttt{\str{K}xLxE/Q/D} and relaxed PEXEL motif \texttt{RxxLxE/Q/D} are also seen occasionally \cite{Elsworth2014}. It is known that the amino acid sequence cleaved after the leucine residue in the parasite ER \cite{Goldberg2010} although how the PEXEL motif targets the protein for export remains unclear. The \pf protein Plasmepsin V has been shown to cleave a subset of PEXEL-carrying proteins \cite{Boddey2013}.

% Where do pexel proteins go next?

\subsection*{PEXEL Negative Exported Proteins (PNEPs)}

In addition to exported PEXEL proteins, several PEXEL Negative proteins have been identified as exportome components using transcription profiling \cite{Heiber2013}.

\subsection{Protein Structure Prediction}

\subsubsection*{Disorder Prediction (metaPrDOS}

\subsubsection*{Coiled Coil Prediction (Coils)}

\subsubsection*{Transmembrane Prediction (TMHMM)}

\subsubsection*{Combined Approaches (Phyre2 and InterPro)}


\section{Software Implementation}
%2000 words
\subsection{Automating Sequence Submission}
\subsection{Collating output formats}
\subsection{Visualisation}
\section{Limitations and Further Work}
%300 words
%------------------------------------------------
\phantomsection
\section*{Acknowledgments}

\addcontentsline{toc}{section}{Acknowledgments} % Adds this section to the table of contents

%----------------------------------------------------------------------------------------
%	REFERENCE LIST
%----------------------------------------------------------------------------------------
\phantomsection
\bibliographystyle{unsrt}
\bibliography{references}

%----------------------------------------------------------------------------------------

\end{document}